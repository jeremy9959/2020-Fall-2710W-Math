% Options for packages loaded elsewhere
\PassOptionsToPackage{unicode}{hyperref}
\PassOptionsToPackage{hyphens}{url}
%
\documentclass[
]{article}
\usepackage{lmodern}
\usepackage{amssymb,amsmath}
\usepackage{ifxetex,ifluatex}
\ifnum 0\ifxetex 1\fi\ifluatex 1\fi=0 % if pdftex
  \usepackage[T1]{fontenc}
  \usepackage[utf8]{inputenc}
  \usepackage{textcomp} % provide euro and other symbols
\else % if luatex or xetex
  \usepackage{unicode-math}
  \defaultfontfeatures{Scale=MatchLowercase}
  \defaultfontfeatures[\rmfamily]{Ligatures=TeX,Scale=1}
\fi
% Use upquote if available, for straight quotes in verbatim environments
\IfFileExists{upquote.sty}{\usepackage{upquote}}{}
\IfFileExists{microtype.sty}{% use microtype if available
  \usepackage[]{microtype}
  \UseMicrotypeSet[protrusion]{basicmath} % disable protrusion for tt fonts
}{}
\makeatletter
\@ifundefined{KOMAClassName}{% if non-KOMA class
  \IfFileExists{parskip.sty}{%
    \usepackage{parskip}
  }{% else
    \setlength{\parindent}{0pt}
    \setlength{\parskip}{6pt plus 2pt minus 1pt}}
}{% if KOMA class
  \KOMAoptions{parskip=half}}
\makeatother
\usepackage{xcolor}
\IfFileExists{xurl.sty}{\usepackage{xurl}}{} % add URL line breaks if available
\IfFileExists{bookmark.sty}{\usepackage{bookmark}}{\usepackage{hyperref}}
\hypersetup{
  hidelinks,
  pdfcreator={LaTeX via pandoc}}
\urlstyle{same} % disable monospaced font for URLs
\setlength{\emergencystretch}{3em} % prevent overfull lines
\providecommand{\tightlist}{%
  \setlength{\itemsep}{0pt}\setlength{\parskip}{0pt}}
\setcounter{secnumdepth}{-\maxdimen} % remove section numbering
\ifluatex
  \usepackage{selnolig}  % disable illegal ligatures
\fi

\author{}
\date{}

\begin{document}

\hypertarget{solutions-to-selected-homework-from-chapter-12}{%
\section{Solutions to selected homework from chapter
12}\label{solutions-to-selected-homework-from-chapter-12}}

\hypertarget{section}{%
\subsection{12.5.10}\label{section}}

\textbf{Problem:} Consider \(f:\mathbb{N}\to\mathbb{Z}\) defined by \[
f(n) = \frac{(-1)^{n}(2n-1)+1}{4}.
\] This function is bijective by a previous exercise. Find its inverse.

\textbf{Solution:} Because of the term \((-1)^n\), the values of this
function depend heavily on whether or not \(n\) is even. If \(n\)
\emph{is} even then we can write \(n=2k\) and we have \[
f(n)=\frac{2n-1+1}{4} = \frac{n}{2}=k.
\] In other words, if \(n\) is even, then \(f(n)\) is \(n/2\). Since
\(n\) is an even natural number, \(n/2\) is a positive integer greater
than or equal to \(1\).

To construct the inverse of this part of the function, we can start with
a positive integer \(k\ge 1\) and define \(f^{-1}(k)=2k\).

If \(n\) is odd, then we can write \(n=2k+1\) and we have \[
f(n)=\frac{(1-2n)+1}{4}=\frac{2-2n}{4}=\frac{2-4k-2}{4}=-k
\] so if \(n\) is odd then \(f(n)\) is \((1-n)/2\). Since \(n\) is a
natural number, \((1-n)/2\) will be a non-positive integer. So to
reverse this part of the function, given a non-positive integer \(k\),
we can let \(n=1-2k\). This will be a positive odd natural number.

So putting the two parts together, we have \[
f^{-1}(k) = \begin{cases} 2k & k>0 \\ 1-2k & k\le 0\end{cases}
\]

\hypertarget{problem-12.6.6}{%
\subsection{Problem 12.6.6}\label{problem-12.6.6}}

\textbf{Problem:} Given a function \(f:A\to B\) and a subset
\(Y\subset B\), is \(f(f^{-1}(Y))=Y\) always true? Prove or give a
counterexample.

\textbf{Solution:} Notice that \(f^{-1}(Y)\) is the subset of \(A\)
consisting of elements \(a\in A\) such that \(f(a)\in Y\). So
\(f(f^{-1}(Y))\subset Y\). The question is whether \(f(f^{-1}(Y))\)
might be \emph{smaller} than all of \(Y\); and indeed it can. Here is a
simple example. Let \(A=\{0\}\) and \(B=\{0,1\}\). Suppose that
\(f=\{(0,0)\}\subset A\times B\) and let \(Y=B\). Then \(f^{-1}(B)=A\)
since \(f^{-1}(0)=0\). But \(f(A)=\{0\}\subset B\) which is smaller than
all of \(B\).

\hypertarget{problem-12.6.8}{%
\subsection{Problem 12.6.8}\label{problem-12.6.8}}

\textbf{Problem:} Given a function \(f:A\to B\) and subsets
\(W,X\subset A\), then \(f(W\cap X)=f(W)\cap f(X)\) is \emph{false} in
general. Give a counterexample.

\textbf{Solution:} Suppose \(A=\{0,1\}\) and \(B=\{0\}\). Suppose that
\(f=\{(0,0),(1,0)\}\subset A\times B\). Let \(W=\{0\}\) and \(X=\{1\}\).
Then \(W\cap X=\emptyset\) so \(f(W\cap X)=\emptyset\). On the other
hand, \(f(W)=B\) and \(f(X)=B\) so \(f(W)\cap f(X)=B\not=\emptyset\).

Notice that you can find a counterexample in both of these cases using
very small sets.

\hypertarget{problem-12.6.12}{%
\subsection{Problem 12.6.12}\label{problem-12.6.12}}

\textbf{Problem:} Consider \(f:A\to B\). Prove that \(f\) is injective
if and only if \(X=f^{-1}(f(X))\) for all \(X\subset A\). Prove that
\(f\) is surjective if and only if \(f(f^{-1}(Y))=Y\) for all
\(Y\subset B\).

\textbf{Solution:} Let's first notice that \(X\subset f^{-1}(f(X))\) for
any \(X\) and any \(f\). To see this, suppose \(x\in X\). Then
\(f(x)\in f(X)\). Since \(f^{-1}(f(X))\) is the set of all elements
\(a\) of \(A\) such that \(f(a)\in X\), we have \(x\in f^{-1}(f(X))\).
Therefore \(X\subset f^{-1}(f(X))\). Suppose that there exists
\(u\in f^{-1}(f(X))\) such that \(u\not\in X\). Then \(f(u)\in f(X)\) so
\(f(u)=f(x)\) for some \(x\in X\), and \(u\not=x\) since \(u\not\in X\).
Therefore \(f\) is not injective. Thus we've proven that if
\(X\not=f^{-1}(f(X))\) then \(f\) is not injective.

Now suppose \(f\) is not injective, so there are two elements \(a\) and
\(a'\) in \(A\) with \(f(a)=f(a')\). Let \(X=\{a\}\). Then
\(f(a)\in f(X)\), so \(a'\in f^{-1}(f(X))\), and therefore
\(f^{-1}(f(X))\not=X\) for this particular \(X\). So if \(f\) is not
injective then there is an \(X\) with \(X\not=f^{-1}(f(X))\).

The surjectivity argument is similar, although everything is switched
around. First notice that, for any \(f\), \(f(f^{-1}(Y))\subset Y\).
This is because if \(a\in f^{-1}(Y)\), then \(f(a)\in Y\) by definition.
So suppose there is an element \(y\) of \(Y\) that is not in
\(f(f^{-1}(Y))\). If there were an \(x\) with \(f(x)=y\), then \(x\)
would be in \(f^{-1}(Y)\), and so \(y=f(x)\) would be in
\(f(f^{-1}(Y))\). So there is no such \(x\), and therefore \(f\) is not
surjective.

On the other hand, suppose \(f\) is not surjective. Then there is a
\(b\in B\) for which there is no \(a\in A\) with \(f(a)=b\). Let
\(Y=\{b\}\). Then \(f^{-1}(Y)=\emptyset\) and
\(f(f^{-1}(Y))=f(\emptyset)=\emptyset\). Thus if \(f\) is not
surjective, there is a subset \(Y\) for which \(f(f^{-1}(Y))\not=Y\).

\end{document}
