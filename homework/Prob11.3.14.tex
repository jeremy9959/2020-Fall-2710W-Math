% Options for packages loaded elsewhere
\PassOptionsToPackage{unicode}{hyperref}
\PassOptionsToPackage{hyphens}{url}
%
\documentclass[
]{article}
\usepackage{lmodern}
\usepackage{amssymb,amsmath}
\usepackage{ifxetex,ifluatex}
\ifnum 0\ifxetex 1\fi\ifluatex 1\fi=0 % if pdftex
  \usepackage[T1]{fontenc}
  \usepackage[utf8]{inputenc}
  \usepackage{textcomp} % provide euro and other symbols
\else % if luatex or xetex
  \usepackage{unicode-math}
  \defaultfontfeatures{Scale=MatchLowercase}
  \defaultfontfeatures[\rmfamily]{Ligatures=TeX,Scale=1}
\fi
% Use upquote if available, for straight quotes in verbatim environments
\IfFileExists{upquote.sty}{\usepackage{upquote}}{}
\IfFileExists{microtype.sty}{% use microtype if available
  \usepackage[]{microtype}
  \UseMicrotypeSet[protrusion]{basicmath} % disable protrusion for tt fonts
}{}
\makeatletter
\@ifundefined{KOMAClassName}{% if non-KOMA class
  \IfFileExists{parskip.sty}{%
    \usepackage{parskip}
  }{% else
    \setlength{\parindent}{0pt}
    \setlength{\parskip}{6pt plus 2pt minus 1pt}}
}{% if KOMA class
  \KOMAoptions{parskip=half}}
\makeatother
\usepackage{xcolor}
\IfFileExists{xurl.sty}{\usepackage{xurl}}{} % add URL line breaks if available
\IfFileExists{bookmark.sty}{\usepackage{bookmark}}{\usepackage{hyperref}}
\hypersetup{
  hidelinks,
  pdfcreator={LaTeX via pandoc}}
\urlstyle{same} % disable monospaced font for URLs
\setlength{\emergencystretch}{3em} % prevent overfull lines
\providecommand{\tightlist}{%
  \setlength{\itemsep}{0pt}\setlength{\parskip}{0pt}}
\setcounter{secnumdepth}{-\maxdimen} % remove section numbering

\author{}
\date{}

\begin{document}

\hypertarget{problem-11.3.14}{%
\section{Problem 11.3.14}\label{problem-11.3.14}}

\renewcommand{\subset}{\subseteq}

\textbf{Problem:} Suppose that \(R\) is a reflexive and symmetric
relation on a finite set \(A\). Define a relation \(S\) on \(A\) by
declaring that \(xSy\) if and only if, for some \(n\in\mathbb{N}\),
there are elements \(x_1,\ldots, x_n\in A\) satisfying
\(xRx_1,x_1Rx_2,\ldots, x_{n-1}Rx_{n}\) and \(x_{n}Ry\).

\begin{enumerate}
\def\labelenumi{\arabic{enumi}.}
\tightlist
\item
  Show that \(S\) is an equivalence relation.
\item
  Show that \(R\subseteq S\).
\item
  Show that \(S\) is the unique smallest equivalence relation on \(A\)
  containing \(R\).
\end{enumerate}

\textbf{Discussion:} The idea of this problem is to show that, given a
reflexive and symmetric relation \(R\) which isn't necessarily
transitive, you can make a relation that is consistent with the original
relation but which \emph{is} transitive. The way you do this is to add
in all the ordered pairs \((x,y)\) that \emph{should} be related if the
relation were transitive. For example, if \((a,b)\in R\) and
\((b,c)\in R\), and \(R\) were transitive, then \((a,c)\) should be in
\(R\). In making our transitive relation, then, we keep all the ordered
pairs in \(R\) and also add \((a,c)\).

The condition that we declare \(xSy\) -- meaning that we include
\((x,y)\) in \(S\) if ``there exist \(x_1,\ldots, x_n\in A\) so that
\(xRx_1, x_1Rx_2,\ldots,x_{n-1}Rx_{n},x_{n}Ry\)'' expresses in formal
terms the idea that we include \((x,y)\) in \(S\) if \(x\) and \(y\)
\emph{should} be related if \(R\) \emph{were} transitive.

In particular, suppose \(R\) were in fact already transitive. Then if
there were a sequence of \(x_i\) as above, the transitive property would
force \((x,y)\) to already be in \(R\). So if \(R\) is transitive, then
the \(S\) constructed in this problem would already be \(R\).

\textbf{Proof:} First we prove that \(S\) is an equivalence relation. We
must show that \(S\) is reflexive, symmetric, and transitive.

\emph{\(S\) is reflexive.} Given \(x\in A\), let \(x_1=x\) and \(y=x\).
Then because \(R\) is reflexive we know that \(xRx_1\) and \(x_1Ry\). So
\(x_1\) is a sequence of length one that meets the condition for \(xSx\)
to be true.

\emph{\(S\) is symmetric.} Let \(x,y\in A\) and suppose \(xSy\). Then
there is a sequence \(x_1,\ldots, x_n\) so that \[
xRx_1,\ldots, x_nRy
\] as in the defining property for \(S\). Since \(R\) is symmetric, we
can reverse all of these ordered pairs to obtain a sequence \[
yRx_n,\ldots, x_1Rx.
\] If we renumber the sequence \(x_1,\ldots, x_n\) in reverse order,
with \(x'_i=x_{n-i+1}\) for \(i=1,\ldots, n\), then we have a sequence
\[
yRx'_1,\ldots, x'_nRx
\] and therefore \(ySx\).

\emph{\(S\) is transitive.} Let \(x,y,z\in A\) and suppose \(xSy\) and
\(ySz\). Then we have sequences \(x_1,\ldots, x_n\) and
\(x'_1,\ldots, x'_m\) so that \[
xRx_1,\ldots, x_nRy
\] and \[
yRx'_1,\ldots, x'_nRz.
\] If we combine the two sequences into a long sequence of length
\(n+m\) where \(x''_i=x_i\) for \(i=1,\ldots, n\) and \(x''_i=x'_{i-n}\)
then we have \[
xRx''_1,\ldots, xRx''_n,xRx''_{n+1},\ldots, x''_{n+m}Rz
\] and so \(xSz\). Therefore \(S\) is transitive.

Next we must show that \(R\subseteq S\). In other words, we must show
that if \(x\) and \(y\) are in \(A\), and \(xRy\), then \(xSy\). For
this, make a sequence where \(x_1=x\) and \(x_2=y\). Then \(xRx_1\)
since \(R\) is reflexive, \(x_1Rx_2\) by hypothesis, and \(x_2Ry\) by
reflexivity again. Therefore \[
xRx_1, x_1Rx_2, x_2Ry
\] gives a sequence that tells us that \(xSy\).

Finally, we need to show that \(S\) is the \emph{unique smallest
equivalence relation on \(A\) containing \(R\)}. Here, \emph{smallest}
means that any other equivalence relation that contains \(R\) also
contains \(S\). In other words, if you want to make \(R\) transitive,
the very least you can do is add the relations that create \(S\). So we
must show that if \(T\) is an equivalence relation that contains \(R\),
then \(S\subseteq T\).

Suppose therefore that \((x,y)\in S\). This means that there is a
sequence \(x_1,\ldots, x_n\) so that \[
xRx_1,\ldots, x_nRy
\] as in the defining property for \(S\). Since \(R\subseteq T\), we
have \[
xTx_1,\ldots, x_nTy
\] and since \(T\) is transitive, this means that \(xTy\). Therefore
\((x,y)\in T\) and we have shown that \(S\subseteq T\).

Finally, we must show that \(S\) is the \emph{unique} smallest
equivalence relation. This means that if \(S\) and \(S'\) are two
equivalence relations containing \(R\), and having the property that, if
\(T\) is another equivalence relation containing \(R\), then
\(S\subseteq T\) and also \(S'\subseteq T\), then \(S=S'\). But if \(S\)
has this property, it means that \(S\subseteq S'\) since \(S'\) is an
equivalence relation; and if \(S'\) has this property it means that
\(S'\subseteq S\). Therefore indeed \(S=S'\).

\end{document}
