% Options for packages loaded elsewhere
\PassOptionsToPackage{unicode}{hyperref}
\PassOptionsToPackage{hyphens}{url}
%
\documentclass[
]{article}
\usepackage{lmodern}
\usepackage{amssymb,amsmath}
\usepackage{ifxetex,ifluatex}
\ifnum 0\ifxetex 1\fi\ifluatex 1\fi=0 % if pdftex
  \usepackage[T1]{fontenc}
  \usepackage[utf8]{inputenc}
  \usepackage{textcomp} % provide euro and other symbols
\else % if luatex or xetex
  \usepackage{unicode-math}
  \defaultfontfeatures{Scale=MatchLowercase}
  \defaultfontfeatures[\rmfamily]{Ligatures=TeX,Scale=1}
\fi
% Use upquote if available, for straight quotes in verbatim environments
\IfFileExists{upquote.sty}{\usepackage{upquote}}{}
\IfFileExists{microtype.sty}{% use microtype if available
  \usepackage[]{microtype}
  \UseMicrotypeSet[protrusion]{basicmath} % disable protrusion for tt fonts
}{}
\makeatletter
\@ifundefined{KOMAClassName}{% if non-KOMA class
  \IfFileExists{parskip.sty}{%
    \usepackage{parskip}
  }{% else
    \setlength{\parindent}{0pt}
    \setlength{\parskip}{6pt plus 2pt minus 1pt}}
}{% if KOMA class
  \KOMAoptions{parskip=half}}
\makeatother
\usepackage{xcolor}
\IfFileExists{xurl.sty}{\usepackage{xurl}}{} % add URL line breaks if available
\IfFileExists{bookmark.sty}{\usepackage{bookmark}}{\usepackage{hyperref}}
\hypersetup{
  hidelinks,
  pdfcreator={LaTeX via pandoc}}
\urlstyle{same} % disable monospaced font for URLs
\setlength{\emergencystretch}{3em} % prevent overfull lines
\providecommand{\tightlist}{%
  \setlength{\itemsep}{0pt}\setlength{\parskip}{0pt}}
\setcounter{secnumdepth}{-\maxdimen} % remove section numbering

\author{}
\date{}

\begin{document}

\hypertarget{problem-11.4.4}{%
\section{Problem 11.4.4}\label{problem-11.4.4}}

\renewcommand{\subset}{\subseteq}

\textbf{Problem:} Suppose that \(P\) is a partition of a set \(A\).
Define a relation \(R\) on \(A\) by declaring that \(xRy\) if and only
if \(x,y\in X\) for some set \(X\in P\). Prove that

\begin{itemize}
\tightlist
\item
  \(R\) is an equivalence relation on \(A\).
\item
  \(P\) is the set of equivalence classes of \(R\).
\end{itemize}

\textbf{Discussion:} Remember that a partition \(P\) of \(A\) is a set
of subsets \(X\subseteq A\) such that each element of \(A\) belongs to
exactly one of the subsets and the union of all of the subsets is \(A\).

In other words, the partition divides \(A\) up into a family of disjoint
subsets that together cover all of \(A\).

\textbf{Proof:} An ordered pair \((x,y)\) belongs to \(R\) if and only
if \(x\) and \(y\) belong to the same set \(X\) of the partition \(P\).
We need to show that this definition yields a relation that is
reflexive, symmetric, and transitive.

\emph{\(R\) is reflexive.} By definition, \((x,x)\) belongs to \(R\) if
and only if \(x\) belongs to the same set \(X\) as \(x\). This is
clearly true since there is only one element \(x\) involved.

\emph{\(R\) is symmetric.} Suppose that \((x,y)\in R\). This means that
\(x\) and \(y\) belong to the same set \(X\in P\). This condition
doesn't care about the order of \(x\) and \(y\), so \(R\) is symmetric.

\emph{\(R\) is transitive.} Suppose that \((x,y)\in R\) and
\((y,z)\in R\). This means there are two sets \(X_1,X_2\in P\) so that
\(x,y\in X_1\) and \(y,z\in X_2\). But in this case,
\(y\in X_1\cap X_2\), and since the elements of a partition are either
disjoint or equal, we must have \(X_1=X_2=X\). Therefore \(x,z\in X\) so
\((x,z)\in R\).

Next, we have to prove that the equivalence classes of the relation
\(R\) are exactly the sets of the partition \(P\). This is a question
about \emph{equality of sets} -- if we let \(Q\) be the set of
equivalence classes of \(R\), we are trying to show that \(Q=P\). As
usual with such proofs, we need to prove that \(Q\subseteq P\) and
\(P\subseteq Q\). In other words, we must show both of the following:

\begin{itemize}
\tightlist
\item
  every equivalence class of \(R\) is an element of the set \(P\).
\item
  every element of the set \(P\) is an equivalence class of \(R\).
\end{itemize}

For the first, let \(a\in A\) and let \[
[a]=\{x\in A: xRa\}.
\] Let \(X\) be the element of \(P\) containing \(a\). (We know there is
only one such \(X\) since \(P\) is a partition). If \(z\in X\), then
\(zRa\), so \(z\in [a]\). Therefore \(X\subseteq[a]\). If \(z\in [a]\),
then \(zRa\), so \(z\) belongs to \(X\), so \([a]\subseteq X\).
Therefore \([a]=X\). We have shown that every equivalence class is one
of the sets of the partition \(P\).

Now let \(X\in P\) be one of the elements of the partition. Choose
\(a\in X\) and let \([a]\) be the equivalence class of \(a\) under
\(R\). If \(z\in X\), then \(zRa\) so \(z\in [a]\); therefore
\(X\subseteq[a]\). If \(z\in [a]\), then \(zRa\) which means \(z\) and
\(a\) belong to the same set of the partition; since \(a\in X\), we have
\(z\in X\). Therefore \([a]\subseteq X\). Thus \([a]=X\) and so every
\(X\in P\) is one of the equivalence classes.

Thus the two partitions (the one given by \(P\), and the one given by
equivalence classes of \(R\)) are actually the same.

\end{document}
