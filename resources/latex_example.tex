\documentclass[12pt]{article}
\usepackage{amsmath,amsthm,amsfonts,amssymb}
\usepackage{hyperref}
\usepackage{setspace}
\usepackage[margin=1in]{geometry}

\theoremstyle{definition}
\newtheorem*{prop}{Proposition}
\newtheorem*{prob}{Problem}

\newcommand{\exer}[1]{\begin{center}
    \medskip
    \textbf{Exercise #1}
    \medskip\hrule\medskip
    \end{center}}
    

\newcommand{\R}{\mathbb{R}}
\newcommand{\Z}{\mathbb{Z}}

    
\setlength{\parindent}{15pt}
\setstretch{1.5}


\title{Math 2710}
\author{Jeremy Teitelbaum}
\date{\today}

\begin{document}
\maketitle

\paragraph{Problem 2.1}
Prove that the sum of even numbers is even.


\begin{proof}
If $x$ is even, it means that there is a $y$ so that $x=2y$.
Suppose that $x_1$ and $x_2$ are even, and that $x_1=2y_1$ and $x_2=2y_2$.  Then
$$ (x_1+x_2) = (2y_1 + 2y_2) =2 (y_1+y_2)
$$
so $x_1+x_2$ is also even. 
\end{proof}


\paragraph{Problem 1.5.2}
Show that $\sim (P\lor Q)=(\sim P \land \sim Q)$.

\begin{proof}
Here is a proof by truth table.
\medskip

\begin{tabular}{|c|c|c|c|c|c|c|}
\hline
$P$ & $Q$ & $P\lor Q$ &$\lnot (P\lor Q)$ & $\lnot P$&$\lnot Q$ & $\lnot P \land \lnot Q$\cr
\hline
T & T & T & F & F & F & F \cr
T & F & T & F & F & T & F \cr
F & T & T & F & T & F & F \cr
F & F & F & T & T & T & T \cr
\hline
\end{tabular}

\medskip\noindent
Notice that the columns for $\lnot (P\lor Q)$ and $\lnot P \land\lnot Q$ are the same. 
\end{proof}


\end{document}