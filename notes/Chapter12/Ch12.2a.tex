% Options for packages loaded elsewhere
\PassOptionsToPackage{unicode}{hyperref}
\PassOptionsToPackage{hyphens}{url}
%
\documentclass[
]{article}
\usepackage{lmodern}
\usepackage{amssymb,amsmath}
\usepackage{ifxetex,ifluatex}
\ifnum 0\ifxetex 1\fi\ifluatex 1\fi=0 % if pdftex
  \usepackage[T1]{fontenc}
  \usepackage[utf8]{inputenc}
  \usepackage{textcomp} % provide euro and other symbols
\else % if luatex or xetex
  \usepackage{unicode-math}
  \defaultfontfeatures{Scale=MatchLowercase}
  \defaultfontfeatures[\rmfamily]{Ligatures=TeX,Scale=1}
\fi
% Use upquote if available, for straight quotes in verbatim environments
\IfFileExists{upquote.sty}{\usepackage{upquote}}{}
\IfFileExists{microtype.sty}{% use microtype if available
  \usepackage[]{microtype}
  \UseMicrotypeSet[protrusion]{basicmath} % disable protrusion for tt fonts
}{}
\makeatletter
\@ifundefined{KOMAClassName}{% if non-KOMA class
  \IfFileExists{parskip.sty}{%
    \usepackage{parskip}
  }{% else
    \setlength{\parindent}{0pt}
    \setlength{\parskip}{6pt plus 2pt minus 1pt}}
}{% if KOMA class
  \KOMAoptions{parskip=half}}
\makeatother
\usepackage{xcolor}
\IfFileExists{xurl.sty}{\usepackage{xurl}}{} % add URL line breaks if available
\IfFileExists{bookmark.sty}{\usepackage{bookmark}}{\usepackage{hyperref}}
\hypersetup{
  hidelinks,
  pdfcreator={LaTeX via pandoc}}
\urlstyle{same} % disable monospaced font for URLs
\setlength{\emergencystretch}{3em} % prevent overfull lines
\providecommand{\tightlist}{%
  \setlength{\itemsep}{0pt}\setlength{\parskip}{0pt}}
\setcounter{secnumdepth}{-\maxdimen} % remove section numbering
\ifluatex
  \usepackage{selnolig}  % disable illegal ligatures
\fi

\author{}
\date{}

\begin{document}

\hypertarget{examples-on-injective-surjective-and-bijective-functions}{%
\section{Examples on Injective, Surjective, and Bijective
functions}\label{examples-on-injective-surjective-and-bijective-functions}}

\hypertarget{example-12.4.}{%
\subsection{Example 12.4.}\label{example-12.4.}}

\textbf{Proposition:} The function \(f:\mathbb{R}-\{0\}\to\mathbb{R}\)
defined by the formula \(f(x)=\frac{1}{x}+1\) is injective but not
surjective.

\vfill\eject

\hypertarget{example-12.5.}{%
\subsection{Example 12.5.}\label{example-12.5.}}

\textbf{Proposition:} The function
\(f:\mathbb{R}-\{0\}\to\mathbb{R}-\{1\}\) is injective and surjective
(hence bijective).

\vfill\eject

\hypertarget{example-12.6}{%
\subsection{Example 12.6}\label{example-12.6}}

\textbf{Proposition:} The function
\(g:\mathbb{Z}\times\mathbb{Z}\to\mathbb{Z}\times\mathbb{Z}\) defined by
the formula \(g(m,n) = (m+n,m+2n)\) is both injective and surjective.

\vfill\eject

\end{document}
