% Options for packages loaded elsewhere
\PassOptionsToPackage{unicode}{hyperref}
\PassOptionsToPackage{hyphens}{url}
%
\documentclass[
]{article}
\usepackage{lmodern}
\usepackage{amssymb,amsmath}
\usepackage{ifxetex,ifluatex}
\ifnum 0\ifxetex 1\fi\ifluatex 1\fi=0 % if pdftex
  \usepackage[T1]{fontenc}
  \usepackage[utf8]{inputenc}
  \usepackage{textcomp} % provide euro and other symbols
\else % if luatex or xetex
  \usepackage{unicode-math}
  \defaultfontfeatures{Scale=MatchLowercase}
  \defaultfontfeatures[\rmfamily]{Ligatures=TeX,Scale=1}
\fi
% Use upquote if available, for straight quotes in verbatim environments
\IfFileExists{upquote.sty}{\usepackage{upquote}}{}
\IfFileExists{microtype.sty}{% use microtype if available
  \usepackage[]{microtype}
  \UseMicrotypeSet[protrusion]{basicmath} % disable protrusion for tt fonts
}{}
\makeatletter
\@ifundefined{KOMAClassName}{% if non-KOMA class
  \IfFileExists{parskip.sty}{%
    \usepackage{parskip}
  }{% else
    \setlength{\parindent}{0pt}
    \setlength{\parskip}{6pt plus 2pt minus 1pt}}
}{% if KOMA class
  \KOMAoptions{parskip=half}}
\makeatother
\usepackage{xcolor}
\IfFileExists{xurl.sty}{\usepackage{xurl}}{} % add URL line breaks if available
\IfFileExists{bookmark.sty}{\usepackage{bookmark}}{\usepackage{hyperref}}
\hypersetup{
  hidelinks,
  pdfcreator={LaTeX via pandoc}}
\urlstyle{same} % disable monospaced font for URLs
\setlength{\emergencystretch}{3em} % prevent overfull lines
\providecommand{\tightlist}{%
  \setlength{\itemsep}{0pt}\setlength{\parskip}{0pt}}
\setcounter{secnumdepth}{-\maxdimen} % remove section numbering
\ifluatex
  \usepackage{selnolig}  % disable illegal ligatures
\fi

\author{}
\date{}

\begin{document}

\hypertarget{the-pigeonhole-principle}{%
\subsection{The pigeonhole principle}\label{the-pigeonhole-principle}}

\textbf{The Pigeonhole Principle:} Suppose that \(M>N\) and you put
\(M\) balls in \(N\) boxes. Then at least one box has more than one
ball.

For our purposes we will treat this as an axiom, since:

\begin{itemize}
\tightlist
\item
  it seems obvious, and
\item
  to prove it you have to be very careful about what axioms you are
  relying on.
\end{itemize}

The contrapositive version: Suppose that you put \(M\) balls in \(N\)
boxes, and no box contains more than one ball. Then \(M<N\).

\vfill\eject

\hypertarget{the-pigeonhole-principle-and-functions}{%
\subsection{The pigeonhole principle and
functions}\label{the-pigeonhole-principle-and-functions}}

Suppose we have a function \(F:A\to B\) where \(A\) and \(B\) are finite
sets.

\textbf{Proposition:} If \(|A|>|B|\) then \(F\) is not injective.

\textbf{Proof:} Think of the elements of \(A\) as balls and the elements
of \(B\) as boxes. If \(F(a)=b\), then you put ball \(a\) in box \(b\).
If \(F\) is injective, then by the definition of injectivity, different
balls go in different boxes. Thus no box contains more than one ball.
This implies there are at least as many boxes as balls, so
\(|B|\ge |A|\). This is a contradiction of our assumption that
\(|A|>|B|\), so \(F\) is not injective.

\vfill\eject

\hypertarget{the-pigeonhole-principle-and-functions-2}{%
\subsection{The pigeonhole principle and functions
2}\label{the-pigeonhole-principle-and-functions-2}}

Suppose we have a function \(F:A\to B\) where \(A\) and \(B\) are finite
sets.

\textbf{Proposition:} If \(|A|<|B|\), then \(F\) is not surjective.

\textbf{Proof:} Again think of elements of \(A\) as balls and elements
of \(B\) as boxes, with \(F(a)=b\) meaning you put ball \(a\) in box
\(b\). The pigeonhole principle says that at least one box is empty; in
other words, there is some \(b\) such that there is no \(a\) with
\(F(a)=b\). Thus \(F\) is not surjective.

\vfill\eject

\hypertarget{some-example-applications}{%
\subsection{Some example applications}\label{some-example-applications}}

Example from page 234.

\textbf{Proposition:} Suppose \(A\) is a set of any \(10\) integers
between \(1\) and \(100\). Then there are two subsets \(X\subseteq A\)
and \(Y\subseteq A\) such that the sum of the elements of \(X\) is the
same as the sum of the elements of \(Y\).

\vfill\eject

\hypertarget{problem-12.3.5}{%
\subsection{Problem 12.3.5}\label{problem-12.3.5}}

\textbf{Proposition:} Any set of seven distinct natural numbers contains
a pair of numbers whose sum or difference is divisible by \(10\).

\end{document}
