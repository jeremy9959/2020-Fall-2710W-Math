% Options for packages loaded elsewhere
\PassOptionsToPackage{unicode}{hyperref}
\PassOptionsToPackage{hyphens}{url}
%
\documentclass[
]{article}
\usepackage{lmodern}
\usepackage{amssymb,amsmath}
\usepackage{ifxetex,ifluatex}
\ifnum 0\ifxetex 1\fi\ifluatex 1\fi=0 % if pdftex
  \usepackage[T1]{fontenc}
  \usepackage[utf8]{inputenc}
  \usepackage{textcomp} % provide euro and other symbols
\else % if luatex or xetex
  \usepackage{unicode-math}
  \defaultfontfeatures{Scale=MatchLowercase}
  \defaultfontfeatures[\rmfamily]{Ligatures=TeX,Scale=1}
\fi
% Use upquote if available, for straight quotes in verbatim environments
\IfFileExists{upquote.sty}{\usepackage{upquote}}{}
\IfFileExists{microtype.sty}{% use microtype if available
  \usepackage[]{microtype}
  \UseMicrotypeSet[protrusion]{basicmath} % disable protrusion for tt fonts
}{}
\makeatletter
\@ifundefined{KOMAClassName}{% if non-KOMA class
  \IfFileExists{parskip.sty}{%
    \usepackage{parskip}
  }{% else
    \setlength{\parindent}{0pt}
    \setlength{\parskip}{6pt plus 2pt minus 1pt}}
}{% if KOMA class
  \KOMAoptions{parskip=half}}
\makeatother
\usepackage{xcolor}
\IfFileExists{xurl.sty}{\usepackage{xurl}}{} % add URL line breaks if available
\IfFileExists{bookmark.sty}{\usepackage{bookmark}}{\usepackage{hyperref}}
\hypersetup{
  hidelinks,
  pdfcreator={LaTeX via pandoc}}
\urlstyle{same} % disable monospaced font for URLs
\setlength{\emergencystretch}{3em} % prevent overfull lines
\providecommand{\tightlist}{%
  \setlength{\itemsep}{0pt}\setlength{\parskip}{0pt}}
\setcounter{secnumdepth}{-\maxdimen} % remove section numbering
\ifluatex
  \usepackage{selnolig}  % disable illegal ligatures
\fi

\author{}
\date{}

\begin{document}

\hypertarget{permutations}{%
\section{Permutations}\label{permutations}}

\hypertarget{factorials}{%
\subsection{Factorials}\label{factorials}}

\textbf{Definition 1:} For \(n\in\mathbb{Z}\), \(n \ge 0\), define
\(0!=1\) and \(n!=(1)(2)\cdots(n-1)(n)\). Alternatively, define \(n!\)
for non-negative integers \(n\) by setting \(0!=1\) and \(n!=n(n-1)!\).

\textbf{Proposition:} The number of different lists of length \(n\) made
up of elements from the set \(\{1,2,\ldots, n\}\), \emph{without
repetitions}, is \(n!\).

\vfill\eject

\hypertarget{permutations-1}{%
\subsection{Permutations}\label{permutations-1}}

\textbf{Definition:} Let \(X\) be a set. A permutation of \(X\) is a
list of length \(|X|\) of the elements of \(X\), without repetition.
(\textbf{Note:} There are other definitions of permutations in other
contexts, all related to this one).

\textbf{Examples:}

\vfill\eject

\hypertarget{section}{%
\subsection{}\label{section}}

\textbf{Definition:} Let \(X\) be a set. A \(k\)-permutation of \(X\) is
a list of \(k\) elements of \(X\) without repetition. \(P(n,k)\) is the
number of \(k\) permutations of a set with \(n\) elements.

\textbf{Proposition:} The number \(P(n,k)\) of \(k\)-permutations of a
set with \(n\) elements is \(n(n-1)\cdots (n-k+1)\) or, equivalently \[
P(n,k) = \frac{n!}{(n-k)!}
\]

\textbf{Proof:}

\vfill\eject

\hypertarget{examples-of-k-permutations}{%
\subsection{\texorpdfstring{Examples of
\(k\)-permutations}{Examples of k-permutations}}\label{examples-of-k-permutations}}

(Problem 7, page 84) How many \(9\)-digit numbers can be made from the
digits \(1,2,3,4,5,6,7,8,9\) if repetition is not allowed and all the
odd digits occur first, followed by all the even digits.

\vfill\eject

\hypertarget{section-1}{%
\subsection{}\label{section-1}}

(Problem 15, page 84) In a club of 15 people, there is a president,
vice-president, secretary, and treasurer. In how many different ways can
this be done?

\vfill\eject

\hypertarget{section-2}{%
\subsection{}\label{section-2}}

\end{document}
