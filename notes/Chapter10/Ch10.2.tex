% Options for packages loaded elsewhere
\PassOptionsToPackage{unicode}{hyperref}
\PassOptionsToPackage{hyphens}{url}
%
\documentclass[
]{article}
\usepackage{lmodern}
\usepackage{amssymb,amsmath}
\usepackage{ifxetex,ifluatex}
\ifnum 0\ifxetex 1\fi\ifluatex 1\fi=0 % if pdftex
  \usepackage[T1]{fontenc}
  \usepackage[utf8]{inputenc}
  \usepackage{textcomp} % provide euro and other symbols
\else % if luatex or xetex
  \usepackage{unicode-math}
  \defaultfontfeatures{Scale=MatchLowercase}
  \defaultfontfeatures[\rmfamily]{Ligatures=TeX,Scale=1}
\fi
% Use upquote if available, for straight quotes in verbatim environments
\IfFileExists{upquote.sty}{\usepackage{upquote}}{}
\IfFileExists{microtype.sty}{% use microtype if available
  \usepackage[]{microtype}
  \UseMicrotypeSet[protrusion]{basicmath} % disable protrusion for tt fonts
}{}
\makeatletter
\@ifundefined{KOMAClassName}{% if non-KOMA class
  \IfFileExists{parskip.sty}{%
    \usepackage{parskip}
  }{% else
    \setlength{\parindent}{0pt}
    \setlength{\parskip}{6pt plus 2pt minus 1pt}}
}{% if KOMA class
  \KOMAoptions{parskip=half}}
\makeatother
\usepackage{xcolor}
\IfFileExists{xurl.sty}{\usepackage{xurl}}{} % add URL line breaks if available
\IfFileExists{bookmark.sty}{\usepackage{bookmark}}{\usepackage{hyperref}}
\hypersetup{
  hidelinks,
  pdfcreator={LaTeX via pandoc}}
\urlstyle{same} % disable monospaced font for URLs
\setlength{\emergencystretch}{3em} % prevent overfull lines
\providecommand{\tightlist}{%
  \setlength{\itemsep}{0pt}\setlength{\parskip}{0pt}}
\setcounter{secnumdepth}{-\maxdimen} % remove section numbering
\ifluatex
  \usepackage{selnolig}  % disable illegal ligatures
\fi

\author{}
\date{}

\begin{document}

\hypertarget{strong-induction}{%
\section{Strong induction}\label{strong-induction}}

\hypertarget{strong-induction-1}{%
\subsection{Strong induction}\label{strong-induction-1}}

\textbf{Axiom of Induction:} For all \(n\in \mathbb{N}\), Let \(P(n)\)
be a statement. If \(P(1)\) is true and, for all \(n\in\mathbb{N}\),
\(P(n)\implies P(n+1)\), then \(P(n)\) is true for all
\(n\in\mathbb{N}\).

\emph{Strong} induction changes the hypothesis slightly.

\textbf{Strong Induction:} For all \(n\in\mathbb{N}\), let \(P(n)\) be a
statement. If \(P(1)\) is true and, for all \(n\), the statement \[
P(1)\wedge P(2)\wedge\cdots\wedge P(n)\implies P(n+1)
\] is true, then \(P(n)\) is true for all \(n\in\mathbb{N}\).

This means that if you prove \(P(1)\) true, and then, by assuming
\emph{all} of the preceeding statements \(P(1),P(2),\ldots, P(n)\) true
you can prove \(P(n+1)\) true, then all \(P(n)\) are true.

\vfill\eject

\hypertarget{example.-see-page-187.}{%
\subsection{Example. (See page 187).}\label{example.-see-page-187.}}

\textbf{Proposition:} Any score of \(12\) or higher is possible in a
football game where the scores are either field goals (\(3\) points) or
touchdowns (\(7\) points). Notice that \(11\) is not a possible score,
so \(12\) is the smallest score such that all larger scores are
possible.

Example: \(12\) is possible as \(4\times 3\) field goals; \(13\) is
possible as \(2\times 3\) field goals plus a \(7\) point touchdown;
\(14\) is possible as two touchdowns; and so on.

\textbf{Proof:} Suppose that \(P(n)\) is the statement that `\(n\) is a
possible score.' We know that \(P(12)\), \(P(13)\), and \(P(14)\) are
true.

Our \emph{strong induction} hypothesis is this:

Suppose that \(P(12), P(13), P(14),\ldots, P(n)\) are all true and
\(n\ge 15\). We want to show that this implies that \(P(n+1)\) is true.

Since all \(P(n)\) up to \(n\) are true, \(P(n-2)\) is true by the
inductive hypothesis, and so \(n-2\) is a possible score. But then
\(n+1=(n-2)+3\) is also possible, because it's obtained by however you
get \(n-2\), plus a field goal.

This establishes the proof by strong induction.

Notice that the key step was that we had to ``go back'' more than one
step to find what we needed.

\vfill\eject

\hypertarget{strong-induction-contd}{%
\subsection{Strong induction cont'd}\label{strong-induction-contd}}

Why does strong induction hold? It holds because it can be changed into
regular induction.

Suppose \(P(n)\) is a sequence of statements that satisfy the conditions
of strong induction, so \(P(1)\) is true and \(P(n+1)\) is a consequence
of \emph{all} of the preceeding statements \(P(1),\ldots,P(n)\).

Let \(S(1)=P(1)\), and let \(S(n)=P(1)\wedge P(2)\cdots\wedge P(n)\). We
apply regular induction to the set of statements \(S(n)\).

\begin{itemize}
\tightlist
\item
  So \(S(n)\) is a sequence of statements, and \(S(1)\) is true.\\
\item
  Also, we know that \(S(n)\implies P(n+1)\) by our hypothesis.\\
\item
  But \(S(n)\wedge P(n+1)=S(n+1)\), and since \(S(n)\) is true and
  \(P(n+1)\) is true, so is \(S(n+1)\).\\
\item
  Therefore we've shown that \(S(1)\) is true and
  \(S(n)\implies S(n+1)\) for all \(n\in\mathbb{N}\).\\
\item
  By \emph{regular} induction, \(S(n)\) is true for all \(n\).\\
\item
  But the only way \(S(n)\) is true is if all \(P(j)\) for
  \(1\le j\le n\) are true.\\
\item
  So all \(P(n)\) are also true.
\end{itemize}

\vfill\eject

\end{document}
